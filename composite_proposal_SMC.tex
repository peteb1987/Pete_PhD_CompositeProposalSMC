\documentclass{article}

\usepackage[pdftex]{graphicx}
\usepackage[caption=false,font=footnotesize]{subfig}

\usepackage{amsmath}
\usepackage{amsfonts}
\usepackage{IEEEtrantools}

\usepackage{algorithm}
\usepackage{algorithmic}

\usepackage{harvard}

\usepackage{color}

\graphicspath{{./}}

\newenvironment{meta}[0]{\color{red} \em}{}

\title{Composite Proposals for SMC Samplers}
\author{Pete Bunch}
\date{December 2012}

\begin{document}

\maketitle

\section{The Situation}

Consider a particle filter for estimating $p(x_{1:t} | y_{1:t})$. Let say that the transition density, $p(x_t | x_{t-1})$, is easily sampled and the likelihood, $p(y_t | x_t)$, can be evaluated, but that this is expensive. The optimal proposal density is $p(x_t | x_{1:t-1}, y_t)$. In our example, this cannot be sampled from, nor easily approximated, and is multimodal.

If we simply use the transition density as the importance density (i.e. a bootstrap filter) then we will need many particles to get a good approximation.



\section{Using a Markov Chain}

We could use MCMC to draw a sample from the optimal importance density. We start by sampling from the transition density, then propose a perturbation to this state and accept or reject it in the usual way, targeting the OID. Continue until the chain has converged. The advantage of such a scheme over simply using loads more particles in the PF is if the transition density and likelihood have a poor overlap. The Markov chain can move the particle to the likelihood peak.

This idea has two drawbacks. Firstly, we need to wait for the chain to converge, and we don't know how long that will be. Secondly, the resulting weight will include a $p(y_t | x_{t-1})$ term, which is intractable.



\section{Using a Markov Chain in an SMC Sampler}

The solution to these problems is to use an SMC sampler which targets the extended distribution over all the intermediate variables used in the Markov Chain (in a similar manner to a PMCMC scheme).

For example, for a length-1 chain, the proposal is,
%
\begin{IEEEeqnarray}{rCl}
 p(x_{1:t-1} | y_{1:t-1})  K_1(x_t^{(1)} | x_{t-1}, y_t)  K_2(x_t^{(2)} | x_t^{(1)}, x_{t-1}, y_t)     ,
\end{IEEEeqnarray}
%
where $K_1(x_t^{(1)} | x_{t-1}, y_t)$ is any valid PF proposal density (e.g. the transition density), and
%
\begin{IEEEeqnarray}{rCl}
K_2(x_t^{(2)} | x_t^{(1)}, x_{t-1}, y_t) & = & \alpha^{(2)} q(x_t^{(2)} | x_t^{(1)}, x_{t-1}, y_t) + (1-\alpha^{(2)}) \delta_{x_t^{(1)}}(x_t^{(2)}) \\
\alpha^{(2)} & = & \frac{ p(y_t | x_t^{(2)}) p(x_t^{(2)} | x_{t-1}) q(x_t^{(1)} | x_t^{(2)}, x_{t-1}, y_t) }{ p(y_t | x_t^{(1)}) p(x_t^{(1)} | x_{t-1}) q(x_t^{(2)} | x_t^{(1)}, x_{t-1}, y_t)  }     ,
\end{IEEEeqnarray}
%
which is the density sampled by MH.


\end{document}